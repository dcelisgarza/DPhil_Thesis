\chapter{Future work}
\label{c:future}

\section{EasyDD}

EasyDD is an actively developed research code. Much remains to be done, here we describe some of the more immediate concerns.

\subsection{Modularisation}

There are still functions with numerous parameters. They can be simplified by adding variables that fall within certain categories into structures. Daniel Hortelano Roig has been doing a lot of work in this regard on a branch in his fork of EasyDD. Some has been backported to the current version of EasyDD, namely the surface node sets, which were used in \cref{c:simulations} to set the boundary conditions.

\subsection{New capabilities}

Other members of the Tarleton Group have been working on adding features to EasyDD. All of Bruce Bromage's work has been merged with the main branch. We worked closely together to make this happen before we finished our projects.

Daniel Hortelano Roig has been increasing the scope of modularisation and encapsulation; making further improvements to the FE parts of EasyDD; integrating the HCP mobility law; and increasing the scope of the types of simulations we can perform. Some of these have already been made available in the main branch but most remain on his own divergent branch, they need to be integrated and tested. Daniel's changes are big, but he is very well-versed in good programming practices as well as EasyDD, so it should not be too difficult for him to accomplish this.

Fengxian Liu's work on diffusion and inclusions is on its own highly divergent branch of the code. Fortunately, her branch has many of the updated subroutines, such as Bruce's new BCC mobility law, and many of those described in \cref{c:easydd,c:topology,c:tractions} (though for some she may not have the newest versions). In particular, the new mobility law, matrix regularisation, analytic tractions and improved integrators have been crucial in getting her simulations to work without unrealistically large amounts of cross slip, reasonable time steps, and to reasonably large strains and dislocation densities \cite{fengxian}. She has modified the source code of her version of EasyDD, but she has been mindful of modulirising her changes as much as possible. We have created the infrastructure to integrate her work relatively easily, but it will require her input. Likely the largest stumbling block will be the force calculation, since the inclusions add a new component to the total force.

Haiyang Yu's work on the effects of hydrogen and nanoindentation \cite{YU2018,yu2020simulating,yu2019influence} will pose the largest amount of trouble. Trying to get his simulations to work properly is what uncovered a lot of the problems described and fixed in \cref{c:easydd,c:topology}. Without these fixes, as well as the analytic tractions of \cref{c:tractions}, it would have been impossible for his simulations to progress as far as they did. With every fix, the proverbial can was kicked down the road, and new problems cropped up as his simulations advanced. When bringing these fixes to other group members, our simulations improved, but by far the most affected were Haiyang's. Despite what it might seem from \cref{c:easydd,c:topology}, these fixes were progressive and often partial for a time. So he made different local copies of EasyDD, as well as different copies of individual subroutines at different stages of `completeness', rather than taking advantage of version-controlled branches. The number and magnitude of the changes he made to the source code, the sprawling organisation and lack of version control, as well as the fact that he is no longer in Oxford, will make the task of integrating his work the hardest of them all by quite a margin, the sooner it's done the better.

\subsection{Fixes}

The fix to the surface remeshing described in \cref{s:surfRem}, may not be ideal. It could perhaps be improved by creating a new node label such that those nodes are not accounted for in the slip step calculation and not be projected to pseudo-infinity by the surface remeshing. This would require modifying the surface remeshing and displacement calculation to exclude these from such processes; as well as modifying the Peach-K\"{o}hler and self-forces to exclude them from those calculations, but allowing them to participate in the remote force calculation as they would be in the ``bulk''.

\subsection{Performance improvements}

The remote force calculation was broken by an incompatibility between a new version of \mintinline{matlab}{MATLAB} and \mintinline{CUDA}{CUDA}. This was fixed during this project, but it uses a very suboptimal memory access pattern, it only has a cache hit rate of $1/9$. As a result, it takes about an order of magnitude more dislocations than the parallelised tractions to break even with the serial version on our current hardware. Changing the memory access pattern would increase the GPU memory used by $2\times$, but would allow the GPU to access the memory $9\times$ more efficiently than the current implementation. Since as we cannot remotely approach the memory capacity of our current GPU with our current capabilities, the increased memory footprint is a price we can afford in the name of increased performance. It could bring it more in line to the performance of the numeric tractions.

\subsection{Conclusions}

EasyDD is an actively developed research code. Its functionality is ever expanding, but in a manner like urban sprawl, unplanned and chaotic. The result is a fractured codebase with at least as many diverging branches as contributors. Unfortunately, many of these are not mutually compatible. This has been an obstacle for the Tarleton Group, but EasyDD is now garnering outside eyes. It's therefore imperative that future and past members make an effort to unify their works. It's also extremely important that any additions and improvements make use of standards, guidelines, and techniques such as those described in \cref{s:rse}. This way, the maintainability and usability of EasyDD will improve rather than decay as more people contribute to its features.
\savearabiccounter
% 500 words
