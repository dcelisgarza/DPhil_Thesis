\chapter{Conclusions}
\label{c:conclusions}

Echoing the project outline, much remains to be done in dislocation dynamics modelling.

This project aimed to apply techniques from diverse fields to accelerate, and improve the ease and accuracy with which outstanding materials science problems can be tackled.

In \cref{c:easydd} we outlined how we fixed an integrator that was severely underperforming; fixed a major numerical instability in our mobility laws; overhauled EasyDD to make it more performant, as well as easier to use, maintain and develop.

In \cref{c:topology} we dove into the weeds of the topological operations of dislocation dynamics. We fixed the collision detection; fixed the incorrect ordering of all topological operations; eliminated a huge source of slowdown from the separation of many armed nodes; solved and generalised the problem with dislocations exiting from fixed ends, thus creating slip steps that generate huge stresses when the FEM enforces the boundary conditions.

In \cref{c:tractions} is from a paper that was put on hold to allow for this document to be written. In it, we show the work and results arising from the implementation of exact analytic tractions, and their parallelisation. We show how these are more accurate and preferable over their numeric counterparts.

In \cref{c:simulations} we put it all together into a remarkably simple, yet very good reproduction of experimental observations. We show how emergent features such as stress-strain curves and slip step formation naturally arise from the motion and interaction of dislocations. However, the small strain constraint of dislocation plasticity, and small number of samples, must temper the conclusions that can be drawn from these.

Finally, in \cref{c:future} we outline and offer tentative solutions to problems that remain in EasyDD. We also show an extremely promising alternative path that arose as an accident of circumstance.

Should anything be evident from the present work is that science is evolving. The low-hanging fruit have been thoroughly and painstakingly picked from the tree. What remains are the complexities and intricacies that do not fall neatly in the boxes we have created for them. Phenomena that lie hidden in the shadows cast by those ever-expanding recepticles. In a manner not dissimilar to crystal growth, these boxes keep exapnding until they collide with others. What happens then? Clearly the centre of a crystal grain is well characterised, but what of its boundaries? What happens to a science when it overflows its bounds? The answer is a matter of philosophy. We hope this work contributes---if only a little---to this endeavour.
\savearabiccounter
% 413