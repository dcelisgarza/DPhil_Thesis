\addtocounter{chapter}{-1}
\chapter{Preface}
\label{c:pre}

\section{Notation and abbreviation}
\label{s:nota}

For the sake of clarity the following notation and abbreviation conventions have been used:
\begin{itemize}
	\item GPU: Graphics Processing Unit.
	\item CPU: Central Processing Unit.
	\item Tensors are denoted by sans serif bold italics, $ \tns{T} $.
	\item Matrices are denoted by serif bold roman, $ \mtx{M} $.
	\item Vectors are denoted by serif bold italics, $ \vec{V} $.
	\item Scalars are denoted by serif italics, $ S $.
	\item Operators and special functions are roman, $\exp(x)$, $\det{\mtx{J}}$.
	\item Angles are in radians.
	\item Individual items, be they nodes, surface elements, dislocation segments or indices are denoted by a lower-case subscript to the right, $a_n$.
	\item Ensembles are denoted by an upper-case subscript to the right, $a_N \coloneqq \sum a_n$.
	\item Host variables (CPU variables in parallelisation) are denoted by a roman h-superscript on the left, $\hvar{a}$.
	\item Global device variables (global variables visible only to the GPU) are denoted by a roman d-superscript on the left, $\dvar{a}$.
	\item Thread variables (variables visible only to a GPU thread) are denoted by a roman t-superscript on the left, $\tvar{a}$.
	\item Serial indices/counters are the traditional $i$, $j$, $k$\ldots{}.
	\item Parallel indices are denoted as roman variables, $\rvar{a}$.
	\item Pseudo-code is C-style---row-major order, 0-based indexing---because it provides a more direct translation of the C-code.
\end{itemize}

\section{Typesetting}
\label{s:typeset}

The repository \href{https://github.com/dcelisgarza/latex_templates}{https://github.com/dcelisgarza/latex\_templates} contains the custom document class used to typeset this document. It was compiled with \hologo{XeLaTeX} and \hologo{BibTeX}. We recommend compiling the source with \hologo{XeLaTeX} or \hologo{LuaLaTeX}. Compilation requires \texttt{minted}, \href{https://ctan.org/pkg/minted?lang=en}{https://ctan.org/pkg/minted?lang=en}, and its dependencies.
\section{Diagrams}
\label{s:diag}

All diagrams were drawn with Inkscape: Open Source Scalable Vector Graphics Editor, \href{https://inkscape.org/}{https://inkscape.org/}, ``Draw Freely.''
\savearabiccounter
% 246 words